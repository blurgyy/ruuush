%%%%%%%%%%%%%%%%%%%%%%%%%%%%%%%%%%%%%%%%%%%%%%%%%%%%%%%%%%%%%%%%%%
% ACM ICPC 2016-2017, NEERC                                      %
% Northern Subregional Contest                                   %
% St Petersburg, October 22, 2016                                %
%%%%%%%%%%%%%%%%%%%%%%%%%%%%%%%%%%%%%%%%%%%%%%%%%%%%%%%%%%%%%%%%%%
% Problem H. Hard Cuts                                           %
%                                                                %
% Original idea         Georgiy Korneev                          %
% Problem statement     Gennady Korotkevich                      %
% Test set              Gennady Korotkevich                      %
%%%%%%%%%%%%%%%%%%%%%%%%%%%%%%%%%%%%%%%%%%%%%%%%%%%%%%%%%%%%%%%%%%
% Problem statement                                              %
%                                                                %
% Author                Gennady Korotkevich                      %
%%%%%%%%%%%%%%%%%%%%%%%%%%%%%%%%%%%%%%%%%%%%%%%%%%%%%%%%%%%%%%%%%%

\begin{problem}{Hard Cuts}{hard.in}{hard.out}{\timeLimit}

% Original idea : Georgiy Korneev
% Text          : Gennady Korotkevich
% Tests         : Gennady Korotkevich

Given a rectangle with integer side lengths, your task is to cut it into the smallest possible number of squares with integer side lengths.

\InputFile

The first line contains a single integer $T$~--- the number of test cases
($1 \le T \le 3600$).
Each of the next $T$ lines contains two integers $w_i$, $h_i$~---
the dimensions of the rectangle
($1 \le w_i, h_i \le 60$; for any $i \ne j$, either $w_i \ne w_j$ or $h_i \ne h_j$).

\OutputFile

For the $i$-th test case, output $k_i$~--- the minimal number of squares,
such that it is possible to cut the $w_i$ by $h_i$ rectangle into $k_i$ squares.
The following $k_i$ lines should contain three integers each:
$x_{ij}$, $y_{ij}$~--- the coordinates of the bottom-left corner of the $j$-th square
and $l_{ij}$~--- its side length
($0 \le x_{ij} \le w_i - l_{ij}$; $0 \le y_{ij} \le h_i - l_{ij}$).
The bottom-left corner of the rectangle has coordinates $(0, 0)$ and
the top-right corner has coordinates $(w_i, h_i)$.

\Example

\begin{example}
\exmp{
3
5 3
5 6
4 4
}{
4
0 0 3
3 0 2
3 2 1
4 2 1
5
0 0 2
0 2 2
0 4 2
2 0 3
2 3 3
1
0 0 4
}%
\end{example}

\begin{center}
\begin{tabular}{ccc}
    \includegraphics[height=4.7cm]{pics/hard.1} &
    \includegraphics[height=4.7cm]{pics/hard.2} &
    \includegraphics[height=4.7cm]{pics/hard.3} \\
    Example case 1 &
    Example case 2 &
    Example case 3
\end{tabular}
\end{center}
\end{problem}

